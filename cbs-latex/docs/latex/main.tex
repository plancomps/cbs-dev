\documentclass[fleqn]{article}
\usepackage{amsmath}
\usepackage{amssymb}
\usepackage[dvipsnames]{xcolor}
\usepackage{parskip}
\usepackage{hyperref}

\newcommand{\KEY}[1]{\textsf{\textit{\color{Gray}#1}}}
\newcommand{\VAR}[1]{\mathit{\small#1}}
\newcommand{\LEX}[1]{\text{\texttt{#1}}}
\newcommand{\LBRACE}{\char`\{}
\newcommand{\RBRACE}{\char`\}}
\newcommand{\STRING}[1]{\text{``\texttt{#1}''}}
\newcommand{\ATOM}[1]{\text{`\texttt{#1}'}}

\newcommand{\NAME}[2][\PLAIN]{#1{Name_#2}{\textsf{\color{Mahogany}#2}}}
\newcommand{\SYN}[2][\PLAIN]{#1{SyntaxName_#2}{\textsf{\color{ForestGreen}#2}}}
\newcommand{\SEM}[2][\PLAIN]{#1{SemanticsName_#2}{\textsf{\color{Blue}#2}}}

\newcommand{\PLAIN}[1]{}
\newcommand{\DECL}[1]{\hypertarget{#1}}
\newcommand{\REF}[1]{\hyperlink{#1}}
\newcommand{\HYPER}[2]{#1{#2}}

\newcommand{\GROUP}[1]{ {\color{Gray}(} #1 {\color{Gray})} }
\newcommand{\PLUS}{^{\text{\bf+}}}
\newcommand{\STAR}{^{\boldsymbol{\ast}}}
\newcommand{\QUERY}{^{\text{\bf?}}}
\newcommand{\PHRASE}[1]{[\![\, #1 \,]\!]}

\newcommand{\RULE}[2]{\frac{#1}{#2}}
\newcommand{\AXIOM}[1]{\begin{aligned}#1\end{aligned}}
\newcommand{\TO}{\mathop{\Rightarrow}}
\newcommand{\TRANS}{\longrightarrow}

\newcommand{\LanguagesSIMPLE}[2]{\href{https://plancomps.github.io/CBS-beta/Languages-beta/SIMPLE/SIMPLE-cbs/SIMPLE/SIMPLE-#1\##2}}
\newcommand{\FunconsFlowing}[1]{\href{https://plancomps.github.io/CBS-beta/Funcons-beta/Computations/Normal/Flowing\##1}}
\newcommand{\FunconsBinding}[1]{\REF{#1}}

\title{CBS-LaTeX}
\author{Peter Mosses}
\date{November 2020}

\begin{document}

\maketitle

\section*{Samples}

The \LaTeX\ markup, embedded in Markdown, is to be generated from plain CBS specifications using Spoofax.
Before implementing it, an editor was used to transform two plain text specifications from CBS-beta into the intended markup.

\begin{quote}
All the syntax and semantics names below are linked to their (local or online) declarations, but links for funcon names are generally omitted.
\end{quote}

\subsection*{Language ``SIMPLE''}

\subsubsection*{Statements}

\begin{alignat*}{2}
\KEY{Syntax} ~
  \VAR{Block} & : \SYN[\DECL]{block} & ~ ::= ~ & \LEX{\LBRACE} ~ \SYN[\REF]{stmts}\QUERY ~ \LEX{\RBRACE}
\\
  \VAR{Stmts} & : \SYN[\DECL]{stmts} & ~ ::= ~ & \SYN[\REF]{stmt} ~ \SYN[\REF]{stmts}\QUERY
\\
  \VAR{Stmt} & : \SYN[\DECL]{stmt} & ~ ::= ~ & \SYN[\REF]{imp-stmt} ~ \mid \SYN[\HYPER{\LanguagesSIMPLE{4-Declarations}}]{vars-decl}
\\
  \VAR{ImpStmt} & : \SYN[\DECL]{imp-stmt} & ~ ::= ~ & \SYN[\REF]{block} \\
    & & ~ \mid ~ & \SYN[\HYPER{\LanguagesSIMPLE{2-Expressions}}]{exp} ~ \LEX{;} \\
    & & ~ \mid ~ & \LEX{if} ~ \LEX{(} ~ \SYN[\HYPER{\LanguagesSIMPLE{2-Expressions}}]{exp} ~ \LEX{)} ~ \SYN[\REF]{block} ~ \GROUP{ \LEX{else} ~ \SYN[\REF]{block} }\QUERY \\
    & & ~ \mid ~ & \LEX{while} ~ \LEX{(} ~ \SYN[\HYPER{\LanguagesSIMPLE{2-Expressions}}]{exp} ~ \LEX{)} ~ \SYN[\REF]{block} \\
    & & ~ \mid ~ & \LEX{for} ~ \LEX{(} ~ \SYN[\REF]{stmt} ~ \SYN[\HYPER{\LanguagesSIMPLE{2-Expressions}}]{exp} ~ \LEX{;} ~ \SYN[\HYPER{\LanguagesSIMPLE{2-Expressions}}]{exp} ~ \LEX{)} ~ \SYN[\REF]{block} \\
    & & ~ \mid ~ & \LEX{print} ~ \LEX{(} ~ \SYN[\HYPER{\LanguagesSIMPLE{2-Expressions}}]{exps} ~ \LEX{)} ~ \LEX{;} \\
    & & ~ \mid ~ & \LEX{return} ~ \SYN[\HYPER{\LanguagesSIMPLE{2-Expressions}}]{exp}\QUERY ~ \LEX{;} \\
    & & ~ \mid ~ & \LEX{try} ~ \SYN[\REF]{block} ~ \LEX{catch} ~ \LEX{(} ~ \SYN[\HYPER{\LanguagesSIMPLE{1-Lexical}}]{id} ~ \LEX{)} ~ \SYN[\REF]{block} \\
    & & ~ \mid ~ & \LEX{throw} ~ \SYN[\HYPER{\LanguagesSIMPLE{2-Expressions}}]{exp} ~ \LEX{;}
\end{alignat*}
% $$
% 
% $$
\begin{align*}
\KEY{Rule} ~
  \PHRASE{ \LEX{if} ~ \LEX{(} ~ \VAR{Exp} ~ \LEX{)} ~ \VAR{Block} } : \SYN[\REF]{stmt} =
  \PHRASE{ \LEX{if} ~ \LEX{(} ~ \VAR{Exp} ~ \LEX{)} ~ \VAR{Block} ~ \LEX{else} ~ \LEX{\LBRACE} ~ \LEX{\RBRACE} }
\end{align*}
% $$
% 
% $$
\begin{align*}
\KEY{Rule} ~
  & \PHRASE{ \LEX{for} ~ \LEX{(} ~ \VAR{Stmt} ~ \VAR{Exp}_1 ~ \LEX{;} ~ \VAR{Exp}_2 ~ \LEX{)} ~
       \LEX{\LBRACE} ~ \VAR{Stmts} ~ \LEX{\RBRACE} } : \SYN[\REF]{stmt} = \\
  & \PHRASE{ \LEX{\LBRACE} ~ \VAR{Stmt} ~
         \LEX{while} ~ \LEX{(} ~ \VAR{Exp}_1 ~ \LEX{)} ~
            \LEX{\LBRACE} ~ \LEX{\LBRACE} ~ \VAR{Stmts} ~ \LEX{\RBRACE} ~ \VAR{Exp}_2 ~ \LEX{;} ~ \LEX{\RBRACE} ~
     \LEX{\RBRACE} }
\end{align*}
% $$
% 
\[ % $$
\KEY{Semantics} ~
  \SEM[\DECL]{exec} \PHRASE{ \_:\SYN[\REF]{stmts} } : \TO \NAME{null-type}
\] % $$
% 
% $$
\begin{align*}
\KEY{Rule} ~
  \SEM[\REF]{exec} \PHRASE{ \LEX{\LBRACE} ~ \LEX{\RBRACE} } = \NAME{null}
\end{align*}
% $$
% 
% $$
\begin{align*}
\KEY{Rule} ~
  \SEM[\REF]{exec} \PHRASE{ \LEX{\LBRACE} ~ \VAR{Stmts} ~ \LEX{\RBRACE} } = \SEM[\REF]{exec} \PHRASE{ \VAR{Stmts} }
\end{align*}
% $$
% 
% $$
\begin{align*}
\KEY{Rule} ~
  \SEM[\REF]{exec} \PHRASE{ \VAR{ImpStmt} ~ \VAR{Stmts} } = 
    \NAME[\HYPER{\FunconsFlowing}]{sequential}(\SEM[\REF]{exec} \PHRASE{ \VAR{ImpStmt} }, \SEM[\REF]{exec} \PHRASE{ \VAR{Stmts} })
\end{align*}
% $$
% 
% $$
\begin{align*}
\KEY{Rule} ~
  \SEM[\REF]{exec} \PHRASE{ \VAR{VarsDecl} ~ \VAR{Stmts} } = 
    \NAME[\HYPER{\FunconsBinding}]{scope}(\SEM[\HYPER{\LanguagesSIMPLE{4-Declarations}}]{declare} \PHRASE{ \VAR{VarsDecl} }, \SEM[\REF]{exec} \PHRASE{ \VAR{Stmts} })
\end{align*}
% $$
% 
% $$
\begin{align*}
\KEY{Rule} ~
  \SEM[\REF]{exec} \PHRASE{ \VAR{VarsDecl} } = \NAME{effect}(\SEM[\HYPER{\LanguagesSIMPLE{4-Declarations}}]{declare} \PHRASE{ \VAR{VarsDecl}})
\end{align*}
% $$
% 
% $$
\begin{align*}
\KEY{Rule} ~
  \SEM[\REF]{exec} \PHRASE{ \VAR{Exp} ~ \LEX{;} } = \NAME{effect}(\SEM[\HYPER{\LanguagesSIMPLE{2-Expressions}}]{rval} \PHRASE{ \VAR{Exp} })
\end{align*}
% $$
% 
% $$
\begin{align*}
\KEY{Rule} ~
  & \SEM[\REF]{exec} \PHRASE{ \LEX{if} ~ \LEX{(} ~ \VAR{Exp} ~ \LEX{)} ~ \VAR{Block}_1 ~ \LEX{else} ~ \VAR{Block}_2 } = \\
  & \NAME{if-else}(\SEM[\HYPER{\LanguagesSIMPLE{2-Expressions}}]{rval} \PHRASE{ \VAR{Exp} }, \SEM[\REF]{exec} \PHRASE{ \VAR{Block}_1 ~ }, \SEM[\REF]{exec} \PHRASE{ \VAR{Block}_2 })
\end{align*}
% $$
% 
% $$
\begin{align*}
\KEY{Rule} ~
  \SEM[\REF]{exec} \PHRASE{ \LEX{while} ~ \LEX{(} ~ \VAR{Exp} ~ \LEX{)} ~ \VAR{Block} } = \NAME{while}(\SEM[\HYPER{\LanguagesSIMPLE{2-Expressions}}]{rval} \PHRASE{ \VAR{Exp} }, \SEM[\REF]{exec} \PHRASE{ \VAR{Block} })
\end{align*}
% $$
% 
% $$
\begin{align*}
\KEY{Rule} ~
\SEM[\REF]{exec} \PHRASE{ \LEX{print} ~ \LEX{(} ~ \VAR{Exps} ~ \LEX{)} ~ \LEX{;} } = \NAME{print}(\SEM[\HYPER{\LanguagesSIMPLE{2-Expressions}}]{rvals} \PHRASE{ \VAR{Exps} })
\end{align*}
% $$
% 
% $$
\begin{align*}
\KEY{Rule} ~
  \SEM[\REF]{exec} \PHRASE{ \LEX{return} ~ \VAR{Exp} ~ \LEX{;} } = \NAME{return}(\SEM[\HYPER{\LanguagesSIMPLE{2-Expressions}}]{rval} \PHRASE{ \VAR{Exp} })
\end{align*}
% $$
% 
% $$
\begin{align*}
\KEY{Rule} ~
  \SEM[\REF]{exec} \PHRASE{ \LEX{return} ~ \LEX{;} } = \NAME{return}(\NAME{null})
\end{align*}
% $$
% 
% $$
\begin{align*}
\KEY{Rule} ~
  & \SEM[\REF]{exec} \PHRASE{ \LEX{try} \VAR{Block}_1 ~ \LEX{catch} \LEX{(} \VAR{Id} \LEX{)} \VAR{Block}_2 } = \\
  & \NAME{handle-thrown}( \\
  & \quad   \SEM[\REF]{exec} \PHRASE{ \VAR{Block}_1 ~ }, \\
  & \quad   \NAME{scope}( \\
  & \quad \quad   \NAME{bind}(\SEM[\HYPER{\LanguagesSIMPLE{1-Lexical}}]{id} \PHRASE{ \VAR{Id} }, \NAME{allocate-initialised-variable}(\NAME{values},\NAME{given})), \\
  & \quad \quad   \SEM[\REF]{exec} \PHRASE{ \VAR{Block}_2 }))
\end{align*}
% $$
% 
% $$
\begin{align*}
\KEY{Rule} ~
  \SEM[\REF]{exec} \PHRASE{ \LEX{throw} ~ \VAR{Exp} ~ \LEX{;} } = \NAME{throw}(\SEM[\HYPER{\LanguagesSIMPLE{2-Expressions}}]{rval} \PHRASE{ \VAR{Exp} })
\end{align*}
% $$

\subsection*{Funcons-beta/Computations/Normal}

\subsubsection*{Binding}

\paragraph*{Contents}

% $$
\begin{alignat*}{2}
  \KEY{Type} ~~    & \NAME{environments}       &~~& \KEY{Alias} ~~ \NAME{envs}
  \\
  \KEY{Datatype} ~~ & \NAME{identifiers}        & & \KEY{Alias} ~~ \NAME{ids}
  \\
  \KEY{Funcon} ~~  & \NAME{identifier-tagged}  & & \KEY{Alias} ~~ \NAME{id-tagged}
  \\
  \KEY{Funcon} ~~  & \NAME{fresh-identifier}
  \\
  \KEY{Entity} ~~  & \NAME{environment}        & & \KEY{Alias} ~~ \NAME{env}
  \\
  \KEY{Funcon} ~~  & \NAME{initialise-binding}
  \\
  \KEY{Funcon} ~~  & \NAME{bind-value}         & & \KEY{Alias} ~~ \NAME{bind}
  \\
  \KEY{Funcon} ~~  & \NAME{unbind}
  \\
  \KEY{Funcon} ~~  & \NAME{bound-directly}
  \\
  \KEY{Funcon} ~~  & \NAME{bound-value}        & & \KEY{Alias} ~~ \NAME{bound}
  \\
  \KEY{Funcon} ~~  & \NAME{closed}
  \\
  \KEY{Funcon} ~~  & \NAME{scope}
  \\
  \KEY{Funcon} ~~  & \NAME{accumulate}
  \\
  \KEY{Funcon} ~~  & \NAME{collateral}
  \\
  \KEY{Funcon} ~~  & \NAME{bind-recursively}
  \\
  \KEY{Funcon} ~~  & \NAME{recursive}
\end{alignat*}
% $$

\[ % $$
\KEY{Meta-variables} ~
  T <: \NAME{values}
\] % $$

\paragraph*{Environments}

% $$
\begin{align*}
  \KEY{Type} ~
  & \NAME{environments} \leadsto \NAME{maps}(\NAME{identifiers}, \NAME{values}\QUERY)
\\
  \KEY{Alias} ~
  & \NAME{envs} = \NAME{environments}
\end{align*}
% $$
% 
An environment represents bindings of identifiers to values.
Mapping an identifier to $ (~) $ represents that its binding is hidden.

Circularity in environments (due to recursive bindings) is represented using
bindings to cut-points called *links*. Funcons are provided for making
declarations recursive and for referring to bound values without explicit
mention of links, so their existence can generally be ignored.
% 
% $$
\begin{align*}
  \KEY{Datatype} ~
  & \NAME{identifiers} ::= \{\_:\NAME{strings}\} \mid \NAME{identifier-tagged}(\_:\NAME{identifiers}, \_:\NAME{maps})
\\
  \KEY{Alias} ~
  & \NAME{ids} = \NAME{identifiers}
\\
  \KEY{Alias} ~
  & \NAME{id-tagged} = \NAME{identifier-tagged}
\end{align*}
% $$
% 
An identifier is either a string of characters, or an identifier tagged with
some value (e.g., with the identifier of a namespace).
% 
\[ % $$
\KEY{Funcon} ~
  \NAME{fresh-identifier} : \TO \NAME{identifiers}
\] % $$
% 
$ \NAME{fresh-identifier} $ computes an identifier distinct from all previously
computed identifiers.
% 
\[ % $$
\KEY{Rule} ~
  \NAME{fresh-identifier} \leadsto \NAME{identifier-tagged}(\STRING{generated}, \NAME{fresh-atom})
\] % $$

\paragraph*{Current bindings}

% $$
\begin{align*}
  \KEY{Entity} ~
  & \NAME{environment}(\_:\NAME{environments}) \vdash \_ \TRANS \_
\\
  \KEY{Alias} ~
  & \NAME{env} = \NAME{environment}
\end{align*}
% $$
% 
The environment entity allows a computation to refer to the current bindings
of identifiers to values.
% 
% $$
\begin{align*}
  \KEY{Funcon} ~
  & \NAME{initialise-binding}(X:\TO T) : \TO T
\\
  & \quad {} \leadsto \NAME{initialise-linking}(\NAME{initialise-generating}(\NAME{closed}(X)))
\end{align*}
% $$
% 
$ \NAME{initialise-binding}(X) $ ensures that $ X $ does not depend on non-local bindings.
It also ensures that the linking entity (used to represent potentially cyclic
bindings) and the generating entity (for creating fresh identifiers) are 
initialised.
% 
% $$
\begin{align*}
  \KEY{Funcon} ~
  & \NAME{bind-value}(I:\NAME{identifiers}, V:\NAME{maps}) : \TO \NAME{environments}
\\
  & \quad {} \leadsto \{ I \mapsto V \}
\\
  \KEY{Alias} ~
  & \NAME{bind} = \NAME{bind-value}
\end{align*}
% $$
% 
$ \NAME{bind-value}(I, X) $ computes the environment that binds only $ I $ to the value
computed by $ X $.
% 
% $$
\begin{align*}
  \KEY{Funcon} ~
  & \NAME{unbind}(I:\NAME{identifiers}) : \TO \NAME{environments}
\\
  & \quad {} \leadsto \{ I \mapsto (~) \}
\end{align*}
% $$
% 
$ \NAME{unbind}(I) $ computes the environment that hides the binding of $ I $.
% 
\[ % $$
\KEY{Funcon} ~
  \NAME{bound-directly}(\_:\NAME{identifiers}) : \TO \NAME{maps}
\] % $$
% 
$ \NAME{bound-directly}(I) $ returns the value to which $ I $ is currently bound, if any,
and otherwise fails.

$ \NAME{bound-directly}(I) $ does \textit{not} follow links. It is used only in connection with
recursively-bound values when references are not encapsulated in abstractions.
% 
% $$
\begin{align*}
  \KEY{Rule} ~ 
  & \RULE{
    \NAME{lookup}(\rho, I) \leadsto (V:\NAME{maps})
    }{
    \NAME{environment}(\rho) \vdash \NAME{bound-directly}(I:\NAME{identifiers}) \TRANS V
    }
\\
  \KEY{Rule} ~ 
  & \RULE{
    \NAME{lookup}(\rho, I) \leadsto (~)
    }{
    \NAME{environment}(\rho) \vdash \NAME{bound-directly}(I:\NAME{identifiers}) \TRANS \NAME{fail}
    }
\end{align*}
% $$
% 
% $$
\begin{align*}
  \KEY{Funcon} ~
  & \NAME{bound-value}(I:\NAME{identifiers}) : \TO \NAME{maps}
\\
  & \quad {} \leadsto \NAME{follow-if-link}(\NAME{bound-directly}(I))
\\
  \KEY{Alias} ~
  & \NAME{bound} = \NAME{bound-value}
\end{align*}
% $$
% 
$ \NAME{bound-value}(I) $ inspects the value to which $ I $ is currently bound, if any,
and otherwise fails. If the value is a link, $ \NAME{bound-value}(I) $ returns the
value obtained by following the link, if any, and otherwise fails. If the 
inspected value is not a link, $ \NAME{bound-value}(I) $ returns it. 

$ \NAME{bound-value}(I) $ is used for references to non-recursive bindings and to
recursively-bound values when references are encapsulated in abstractions.
\paragraph*{Scope}
% 
\[ % $$
\KEY{Funcon} ~
  \NAME{closed}(X:\TO T) : \TO T
\] % $$
% 
$ \NAME{closed}(X) $ ensures that $ X $ does not depend on non-local bindings.
% 
% $$
\begin{align*}
  \KEY{Rule} ~ 
  & \RULE{
    \NAME{environment}(\NAME{map}(~)) \vdash X \TRANS X'
    }{
    \NAME{environment}(\_) \vdash \NAME{closed}(X) \TRANS \NAME{closed}(X')
    }
\\
  \KEY{Rule} ~
  & \NAME{closed}(V:T) \leadsto V
\end{align*}
% $$
% 
\[ % $$
\KEY{Funcon} ~
  \NAME{scope}(\_:\NAME{environments}, \_:\TO T) : \TO T
\] % $$
% 
$ \NAME{scope}(D,X) $ executes $ D $ with the current bindings, to compute an environment
$ \rho $ representing local bindings. It then executes $ X $ to compute the result,
with the current bindings extended by $ \rho $, which may shadow or hide previous
bindings.

$ \NAME{closed}(\NAME{scope}(\rho, X)) $ ensures that $ X $ can reference only the bindings
provided by $ \rho $.
% 
% $$
\begin{align*}
  \KEY{Rule} ~ 
  & \RULE{
    \NAME{environment}(\NAME{map-override}(\rho_1, \rho_0)) \vdash X \TRANS X'
    }{
    \NAME{environment}(\rho_0) \vdash \NAME{scope}(\rho_1:\NAME{environments}, X) \TRANS \NAME{scope}(\rho_1, X')
    }
\\
  \KEY{Rule} ~
  & \NAME{scope}(\_:\NAME{environments}, V:T) \leadsto V
\end{align*}
% $$
% 
\[ % $$
\KEY{Funcon} ~
  \NAME{accumulate}(\_:(\TO \NAME{environments})\STAR) : \TO \NAME{environments}
\] % $$
% 
$ \NAME{accumulate}(D_1, D_2) $ executes $ D_1 $ with the current bindings, to compute an
environment $ \rho_1 $ representing some local bindings. It then executes $ D_2 $ to
compute an environment $ \rho_2 $ representing further local bindings, with the
current bindings extended by $ \rho_1 $, which may shadow or hide previous
current bindings. The result is $ \rho_1 $ extended by $ \rho_2 $, which may shadow
or hide the bindings of $ \rho_1 $.

$ \NAME{accumulate}(\_, \_) $ is associative, with $ \NAME{map}(~) $ as unit, and extends to any
number of arguments.
% 
% $$
\begin{align*}
  \KEY{Rule} ~ 
  & \RULE{
    D_1 \TRANS D_1'
    }{
    \NAME{accumulate}(D_1, D_2) \TRANS \NAME{accumulate}(D_1', D_2)
    }
\\
  \KEY{Rule} ~
  & \NAME{accumulate}(\rho_1:\NAME{environments}, D_2) \leadsto \NAME{scope}(\rho_1, \NAME{map-override}(D_2, \rho_1))
\\
  \KEY{Rule} ~
  & \NAME{accumulate}(~) \leadsto \NAME{map}(~)
\\
  \KEY{Rule} ~
  & \NAME{accumulate}(D_1) \leadsto D_1
\\
  \KEY{Rule} ~
  & \NAME{accumulate}(D_1, D_2, D\PLUS) \leadsto \NAME{accumulate}(D_1, \NAME{accumulate}(D_2, D\PLUS))
\end{align*}
% $$
% 
% $$
\begin{align*}
  \KEY{Funcon} ~
  & \NAME{collateral}(\rho\STAR:\NAME{environments}\STAR) : \TO \NAME{environments}
\\
  & \quad {} \leadsto \NAME{checked} \NAME{map-unite}(\rho\STAR)
   \end{align*}
% $$
% 
$ \NAME{collateral}(D_1, ...) $ pre-evaluates its arguments with the current bindings,
and unites the resulting maps, which fails if the domains are not pairwise
disjoint.

$ \NAME{collateral}(D_1, D_2) $ is associative and commutative with $ \NAME{map}(~) $ as unit, 
and extends to any number of arguments.

\paragraph*{Recurse}

% $$
\begin{align*}
  \KEY{Funcon} ~
  & \NAME{bind-recursively}(I:\NAME{identifiers}, E:\TO \NAME{maps}) : \TO \NAME{environments}
\\
  & \quad {} \leadsto \NAME{recursive}({I}, \NAME{bind-value}(I, E))
\end{align*}
% $$
% 
$ \NAME{bind-recursively}(I, E) $ binds $ I $ to a link that refers to the value of $ E $, 
representing a recursive binding of $ I $ to the value of $ E $.
Since $ \NAME{bound-value}(I) $ follows links, it should not be executed during the
evaluation of $ E $.
% 
% $$
\begin{align*}
\KEY{Funcon} ~
  & \NAME{recursive}(\VAR{SI}:\NAME{sets}(\NAME{identifiers}), D:\TO \NAME{environments}) : \TO \NAME{environments}
\\
  & \quad ~ {} \leadsto \NAME{re-close}(\NAME{bind-to-forward-links}(\VAR{SI}), D)
\end{align*}
% $$
% 
$ \NAME{recursive}(\VAR{SI}, D) $ executes $ D $ with potential recursion on the bindings of 
the identifiers in the set $ \VAR{SI} $ (which need not be the same as the set of
identifiers bound by $ D $).
% 
% $$
\begin{align*}
  & \KEY{Auxiliary Funcon} ~
\\
  & \quad \NAME{re-close}(M:\NAME{maps}(\NAME{identifiers}, \NAME{links}), D:\TO \NAME{environments}) : \TO \NAME{environments}
\\
  & \quad ~ {} \leadsto \NAME{accumulate}(
    \begin{aligned}[t]
    & \NAME{scope}(M, D), 
    \\
    & \NAME{sequential}(\NAME{set-forward-links}(M), \NAME{map}(~)))
    \end{aligned}
\end{align*}
% $$
% 
$ \NAME{re-close}(M, D) $ first executes $ D $ in the scope $ M $, which maps identifiers
to freshly allocated links. This computes an environment $ \rho $ where the bound
values may contain links, or implicit references to links in abstraction
values. It then sets the link for each identifier in the domain of $ M $ to
refer to its bound value in $ \rho $, and returns $ \rho $ as the result.
% 
% $$
\begin{align*}
  & \KEY{Auxiliary Funcon} ~
\\
  & \quad \NAME{bind-to-forward-links}(\VAR{SI}:\NAME{sets}(\NAME{identifiers})) : \TO \NAME{maps}(\NAME{identifiers}, \NAME{links})
\\
  & \quad ~ {} \leadsto \NAME{map-unite}(
    \begin{aligned}[t]
    & \NAME{interleave-map}(\NAME{bind-value}(\NAME{given}, \NAME{fresh-link}(\NAME{maps})),
    \\
    & \NAME{set-elements}(\VAR{SI})))
    \end{aligned}
\end{align*}
% $$
% 
$ \NAME{bind-to-forward-links}(\VAR{SI}) $ binds each identifier in the set $ \VAR{SI} $ to a
freshly allocated link.
% 
% $$
\begin{align*}
  & \KEY{Auxiliary Funcon} ~
\\
  & \quad \NAME{set-forward-links}(M:\NAME{maps}(\NAME{identifiers}, \NAME{links})) : \TO \NAME{null-type}
\\
  & \quad ~ {} \leadsto \NAME{effect}(\NAME{interleave-map}(
    \begin{aligned}[t]
    & \NAME{set-link}(\NAME{map-lookup}(M, \NAME{given}), \NAME{bound-value}(\NAME{given})),
    \\
    & \NAME{set-elements}(\NAME{map-domain}(M))))
    \end{aligned}
\end{align*}
% $$
% 
For each identifier $ I $ in the domain of $ M $, $ \NAME{set-forward-links}(M) $ sets the 
link to which $ I $ is mapped by $ M $ to the current bound value of $ I $.

\end{document}
