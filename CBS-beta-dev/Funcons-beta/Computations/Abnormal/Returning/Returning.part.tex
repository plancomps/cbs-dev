% 


\begin{center}
\rule{3in}{0.4pt}
\end{center}

\subsubsection{Returning}\hypertarget{returning}{}\label{returning}

\begin{align*}
  [ \
  \KEY{Datatype} \quad & \NAMEREF{returning} \\
  \KEY{Funcon} \quad & \NAMEREF{returned} \\
  \KEY{Funcon} \quad & \NAMEREF{finalise-returning} \\
  \KEY{Funcon} \quad & \NAMEREF{return} \\
  \KEY{Funcon} \quad & \NAMEREF{handle-return}
  \ ]
\end{align*}
\begin{align*}
  \KEY{Meta-variables} \quad
  & \VAR{T} <: \NAMEHYPER{../../../Values}{Value-Types}{values}
\end{align*}
\begin{align*}
  \KEY{Datatype} \quad 
  \NAMEDECL{returning} 
  \ ::= \ & \NAMEDECL{returned}(
                               \_ : \NAMEHYPER{../../../Values}{Value-Types}{values})
\end{align*}
$\SHADE{\NAMEREF{returned}
           (  \VAR{V}\QUERY )}$ is a reason for abrupt termination.

\begin{align*}
  \KEY{Funcon} \quad
  & \NAMEDECL{finalise-returning}(
                       \VAR{X} :  \TO \VAR{T}) 
    :  \TO \VAR{T}  \mid \NAMEHYPER{../../../Values/Primitive}{Null}{null-type} \\&\quad
    \leadsto \NAMEHYPER{../.}{Abrupting}{finalise-abrupting}
               (  \VAR{X} )
\end{align*}
$\SHADE{\NAMEREF{finalise-returning}
           (  \VAR{X} )}$ handles abrupt termination of $\SHADE{\VAR{X}}$ due to
  executing $\SHADE{\NAMEREF{return}
           (  \VAR{V} )}$.

\begin{align*}
  \KEY{Funcon} \quad
  & \NAMEDECL{return}(
                       \VAR{V} : \VAR{T}) 
    :  \TO \NAMEHYPER{../../../Values}{Value-Types}{empty-type} \\&\quad
    \leadsto \NAMEHYPER{../.}{Abrupting}{abrupt}
               (  \NAMEREF{returned}
                       (  \VAR{V} ) )
\end{align*}
$\SHADE{\NAMEREF{return}
           (  \VAR{V} )}$ abruptly terminates all enclosing computations until it is
  handled, then giving $\SHADE{\VAR{V}}$. Note that $\SHADE{\VAR{V}}$ may be $\SHADE{\NAMEHYPER{../../../Values/Primitive}{Null}{null-value}}$.

\begin{align*}
  \KEY{Funcon} \quad
  & \NAMEDECL{handle-return}(
                       \_ :  \TO \VAR{T}) 
    :  \TO \VAR{T} 
\end{align*}
$\SHADE{\NAMEREF{handle-return}
           (  \VAR{X} )}$ first evaluates $\SHADE{\VAR{X}}$. If $\SHADE{\VAR{X}}$ either terminates abruptly for 
  reason $\SHADE{\NAMEREF{returned}
           (  \VAR{V} )}$, or terminates normally with value $\SHADE{\VAR{V}}$, it gives $\SHADE{\VAR{V}}$.

\begin{align*}
  \KEY{Rule} \quad
    & \RULE{
      &  \VAR{X} \xrightarrow{\NAMEHYPER{../.}{Abrupting}{abrupted}(   \  )}_{} 
          \VAR{X}'
      }{
      &  \NAMEREF{handle-return}
                      (  \VAR{X} ) \xrightarrow{\NAMEHYPER{../.}{Abrupting}{abrupted}(   \  )}_{} 
          \NAMEREF{handle-return}
            (  \VAR{X}' )
      }
\\
  \KEY{Rule} \quad
    & \RULE{
      &  \VAR{X} \xrightarrow{\NAMEHYPER{../.}{Abrupting}{abrupted}(  \NAMEREF{returned}
                                                                                  (  \VAR{V} : \NAMEHYPER{../../../Values}{Value-Types}{values} ) )}_{} 
          \VAR{X}'
      }{
      &  \NAMEREF{handle-return}
                      (  \VAR{X} ) \xrightarrow{\NAMEHYPER{../.}{Abrupting}{abrupted}(   \  )}_{} 
          \VAR{V}
      }
\\
  \KEY{Rule} \quad
    & \RULE{
      &  \VAR{X} \xrightarrow{\NAMEHYPER{../.}{Abrupting}{abrupted}(  \VAR{V}' : \mathop{\sim} \NAMEREF{returning} )}_{} 
          \VAR{X}'
      }{
      &  \NAMEREF{handle-return}
                      (  \VAR{X} ) \xrightarrow{\NAMEHYPER{../.}{Abrupting}{abrupted}(  \VAR{V}' )}_{} 
          \NAMEREF{handle-return}
            (  \VAR{X}' )
      }
\\
  \KEY{Rule} \quad
    & \NAMEREF{handle-return}
        (  \VAR{V} : \VAR{T} ) \leadsto 
        \VAR{V}
\end{align*}
% 


