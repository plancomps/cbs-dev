% 



    OUTLINE
  \tableofcontents
\begin{center}
\rule{3in}{0.4pt}
\end{center}

\subsubsection{Sequences of values}\hypertarget{sequences-of-values}{}\label{sequences-of-values}

\begin{align*}
  [ \
  \KEY{Funcon} \quad & \NAMEREF{length} \\
  \KEY{Funcon} \quad & \NAMEREF{index} \\
  \KEY{Funcon} \quad & \NAMEREF{is-in} \\
  \KEY{Funcon} \quad & \NAMEREF{first} \\
  \KEY{Funcon} \quad & \NAMEREF{second} \\
  \KEY{Funcon} \quad & \NAMEREF{third} \\
  \KEY{Funcon} \quad & \NAMEREF{first-n} \\
  \KEY{Funcon} \quad & \NAMEREF{drop-first-n} \\
  \KEY{Funcon} \quad & \NAMEREF{reverse} \\
  \KEY{Funcon} \quad & \NAMEREF{n-of} \\
  \KEY{Funcon} \quad & \NAMEREF{intersperse}
  \ ]
\end{align*}
Sequences of two or more values are not themselves values, nor is the empty
  sequence a value. However, sequences can be provided to funcons as arguments,
  and returned as results. Many operations on composite values can be expressed
  by extracting their components as sequences, operating on the sequences, then
  forming the required composite values from the resulting sequences.

A sequence with elements $\SHADE{\VAR{X}\SUB{1}}$, \ldots{}, $\SHADE{\VAR{X}\SUB{n}}$ is written $\SHADE{\VAR{X}\SUB{1}, \cdots, \VAR{X}\SUB{n}}$.
  A sequence with a single element $\SHADE{\VAR{X}}$ is identified with (and written) $\SHADE{\VAR{X}}$.
  An empty sequence is indicated by the absence of a term.
  Any sequence $\SHADE{\VAR{X}\STAR}$ can be enclosed in parentheses $\SHADE{(  \VAR{X}\STAR )}$, e.g.:
  $\SHADE{(   \  )}$, $\SHADE{(  1 )}$, $\SHADE{(  1, 
                2, 
                3 )}$. Superfluous commas are ignored.

The elements of a type sequence $\SHADE{\VAR{T}\SUB{1}, \cdots, \VAR{T}\SUB{n}}$ are the value sequences $\SHADE{\VAR{V}\SUB{1}, \cdots, \VAR{V}\SUB{n}}$
  where $\SHADE{\VAR{V}\SUB{1} : \VAR{T}\SUB{1}}$, \ldots{}, $\SHADE{\VAR{V}\SUB{n} : \VAR{T}\SUB{n}}$. The only element of the empty type sequence $\SHADE{(   \  )}$
  is the empty value sequence $\SHADE{(   \  )}$.

$\SHADE{(  \VAR{T} )^{\VAR{N}}}$ is equivalent to $\SHADE{\VAR{T}, \cdots, \VAR{T}}$ with $\SHADE{\VAR{N}}$ occurrences of $\SHADE{\VAR{T}}$.
  $\SHADE{(  \VAR{T} )\STAR}$ is equivalent to the union of all $\SHADE{(  \VAR{T} )^{\VAR{N}}}$ for $\SHADE{\VAR{N}}$\textgreater{}=0,
  $\SHADE{(  \VAR{T} )\PLUS}$ is equivalent to the union of all $\SHADE{(  \VAR{T} )^{\VAR{N}}}$ for $\SHADE{\VAR{N}}$\textgreater{}=1, and
  $\SHADE{(  \VAR{T} )\QUERY}$ is equivalent to $\SHADE{\VAR{T}  \mid (   \  )}$.
  The parentheses around $\SHADE{\VAR{T}}$ above can be omitted when they are not needed for
  disambiguation.

(Non-trivial) sequence types are not values, so not included in $\SHADE{\NAMEHYPER{../..}{Value-Types}{types}}$.

\begin{align*}
  \KEY{Meta-variables} \quad
  & \VAR{T}, \VAR{T}' <: \NAMEHYPER{../..}{Value-Types}{values}
\end{align*}
\begin{align*}
  \KEY{Funcon} \quad
  & \NAMEDECL{length}(
                       \_ : \NAMEHYPER{../..}{Value-Types}{values}\STAR) 
    :  \TO \NAMEHYPER{../../Primitive}{Integers}{natural-numbers} 
\end{align*}
$\SHADE{\NAMEREF{length}
           (  \VAR{V}\STAR )}$ gives the number of elements in $\SHADE{\VAR{V}\STAR}$.

\begin{align*}
  \KEY{Rule} \quad
    & \NAMEREF{length}
        (   \  ) \leadsto 
        0
\\
  \KEY{Rule} \quad
    & \NAMEREF{length}
        (  \VAR{V} : \NAMEHYPER{../..}{Value-Types}{values}, 
               \VAR{V}\STAR : \NAMEHYPER{../..}{Value-Types}{values}\STAR ) \leadsto 
        \NAMEHYPER{../../Primitive}{Integers}{natural-successor}
          (  \NAMEREF{length}
                  (  \VAR{V}\STAR ) )
\end{align*}
\begin{align*}
  \KEY{Funcon} \quad
  & \NAMEDECL{is-in}(
                       \_ : \NAMEHYPER{../..}{Value-Types}{values}, \_ : \NAMEHYPER{../..}{Value-Types}{values}\STAR) 
    :  \TO \NAMEHYPER{../../Primitive}{Booleans}{booleans} 
\\
  \KEY{Rule} \quad
    & \NAMEREF{is-in}
        (  \VAR{V} : \NAMEHYPER{../..}{Value-Types}{values}, 
               \VAR{V}' : \NAMEHYPER{../..}{Value-Types}{values}, 
               \VAR{V}\STAR : \NAMEHYPER{../..}{Value-Types}{values}\STAR ) \leadsto 
        \NAMEHYPER{../../Primitive}{Booleans}{or}
          (  \NAMEHYPER{../..}{Value-Types}{is-equal}
                  (  \VAR{V}, 
                         \VAR{V}' ), 
                 \NAMEREF{is-in}
                  (  \VAR{V}, 
                         \VAR{V}\STAR ) )
\\
  \KEY{Rule} \quad
    & \NAMEREF{is-in}
        (  \VAR{V} : \NAMEHYPER{../..}{Value-Types}{values}, 
               (   \  ) ) \leadsto 
        \NAMEHYPER{../../Primitive}{Booleans}{false}
\end{align*}
\paragraph{Sequence indexing}\hypertarget{sequence-indexing}{}\label{sequence-indexing}

\begin{align*}
  \KEY{Funcon} \quad
  & \NAMEDECL{index}(
                       \_ : \NAMEHYPER{../../Primitive}{Integers}{natural-numbers}, \_ : \NAMEHYPER{../..}{Value-Types}{values}\STAR) 
    :  \TO \NAMEHYPER{../..}{Value-Types}{values}\QUERY 
\end{align*}
$\SHADE{\NAMEREF{index}
           (  \VAR{N}, 
                  \VAR{V}\STAR )}$ gives the $\SHADE{\VAR{N}}$th element of $\SHADE{\VAR{V}\STAR}$, if it exists, otherwise $\SHADE{(   \  )}$.

\begin{align*}
  \KEY{Rule} \quad
    & \NAMEREF{index}
        (  1, 
               \VAR{V} : \NAMEHYPER{../..}{Value-Types}{values}, 
               \VAR{V}\STAR : \NAMEHYPER{../..}{Value-Types}{values}\STAR ) \leadsto 
        \VAR{V}
\\
  \KEY{Rule} \quad
    & \RULE{
      & \NAMEHYPER{../../Primitive}{Integers}{natural-predecessor}
          (  \VAR{N} ) \leadsto 
          \VAR{N}'
      }{
      & \NAMEREF{index}
          (  \VAR{N} : \NAMEHYPER{../../Primitive}{Integers}{positive-integers}, 
                 \_ : \NAMEHYPER{../..}{Value-Types}{values}, 
                 \VAR{V}\STAR : \NAMEHYPER{../..}{Value-Types}{values}\STAR ) \leadsto 
          \NAMEREF{index}
            (  \VAR{N}', 
                   \VAR{V}\STAR )
      }
\\
  \KEY{Rule} \quad
    & \NAMEREF{index}
        (  0, 
               \VAR{V}\STAR : \NAMEHYPER{../..}{Value-Types}{values}\STAR ) \leadsto 
        (   \  )
\\
  \KEY{Rule} \quad
    & \NAMEREF{index}
        (  \_ : \NAMEHYPER{../../Primitive}{Integers}{positive-integers}, 
               (   \  ) ) \leadsto 
        (   \  )
\end{align*}
Total indexing funcons:

\begin{align*}
  \KEY{Funcon} \quad
  & \NAMEDECL{first}(
                       \_ : \VAR{T}, \_ : \NAMEHYPER{../..}{Value-Types}{values}\STAR) 
    :  \TO \VAR{T} 
\\
  \KEY{Rule} \quad
    & \NAMEREF{first}
        (  \VAR{V} : \VAR{T}, 
               \VAR{V}\STAR : \NAMEHYPER{../..}{Value-Types}{values}\STAR ) \leadsto 
        \VAR{V}
\end{align*}
\begin{align*}
  \KEY{Funcon} \quad
  & \NAMEDECL{second}(
                       \_ : \NAMEHYPER{../..}{Value-Types}{values}, \_ : \VAR{T}, \_ : \NAMEHYPER{../..}{Value-Types}{values}\STAR) 
    :  \TO \VAR{T} 
\\
  \KEY{Rule} \quad
    & \NAMEREF{second}
        (  \_ : \NAMEHYPER{../..}{Value-Types}{values}, 
               \VAR{V} : \VAR{T}, 
               \VAR{V}\STAR : \NAMEHYPER{../..}{Value-Types}{values}\STAR ) \leadsto 
        \VAR{V}
\end{align*}
\begin{align*}
  \KEY{Funcon} \quad
  & \NAMEDECL{third}(
                       \_ : \NAMEHYPER{../..}{Value-Types}{values}, \_ : \NAMEHYPER{../..}{Value-Types}{values}, \_ : \VAR{T}, \_ : \NAMEHYPER{../..}{Value-Types}{values}\STAR) 
    :  \TO \VAR{T} 
\\
  \KEY{Rule} \quad
    & \NAMEREF{third}
        (  \_ : \NAMEHYPER{../..}{Value-Types}{values}, 
               \_ : \NAMEHYPER{../..}{Value-Types}{values}, 
               \VAR{V} : \VAR{T}, 
               \VAR{V}\STAR : \NAMEHYPER{../..}{Value-Types}{values}\STAR ) \leadsto 
        \VAR{V}
\end{align*}
\paragraph{Homogeneous sequences}\hypertarget{homogeneous-sequences}{}\label{homogeneous-sequences}

\begin{align*}
  \KEY{Funcon} \quad
  & \NAMEDECL{first-n}(
                       \_ : \NAMEHYPER{../../Primitive}{Integers}{natural-numbers}, \_ : (  \VAR{T} )\STAR) 
    :  \TO (  \VAR{T} )\STAR 
\\
  \KEY{Rule} \quad
    & \NAMEREF{first-n}
        (  0, 
               \VAR{V}\STAR : (  \VAR{T} )\STAR ) \leadsto 
        (   \  )
\\
  \KEY{Rule} \quad
    & \RULE{
      & \NAMEHYPER{../../Primitive}{Integers}{natural-predecessor}
          (  \VAR{N} ) \leadsto 
          \VAR{N}'
      }{
      & \NAMEREF{first-n}
          (  \VAR{N} : \NAMEHYPER{../../Primitive}{Integers}{positive-integers}, 
                 \VAR{V} : \VAR{T}, 
                 \VAR{V}\STAR : (  \VAR{T} )\STAR ) \leadsto 
          (  \VAR{V}, 
                 \NAMEREF{first-n}
                  (  \VAR{N}', 
                         \VAR{V}\STAR ) )
      }
\\
  \KEY{Rule} \quad
    & \NAMEREF{first-n}
        (  \VAR{N} : \NAMEHYPER{../../Primitive}{Integers}{positive-integers}, 
               (   \  ) ) \leadsto 
        (   \  )
\end{align*}
\begin{align*}
  \KEY{Funcon} \quad
  & \NAMEDECL{drop-first-n}(
                       \_ : \NAMEHYPER{../../Primitive}{Integers}{natural-numbers}, \_ : (  \VAR{T} )\STAR) 
    :  \TO (  \VAR{T} )\STAR 
\\
  \KEY{Rule} \quad
    & \NAMEREF{drop-first-n}
        (  0, 
               \VAR{V}\STAR : (  \VAR{T} )\STAR ) \leadsto 
        \VAR{V}\STAR
\\
  \KEY{Rule} \quad
    & \RULE{
      & \NAMEHYPER{../../Primitive}{Integers}{natural-predecessor}
          (  \VAR{N} ) \leadsto 
          \VAR{N}'
      }{
      & \NAMEREF{drop-first-n}
          (  \VAR{N} : \NAMEHYPER{../../Primitive}{Integers}{positive-integers}, 
                 \_ : \VAR{T}, 
                 \VAR{V}\STAR : (  \VAR{T} )\STAR ) \leadsto 
          \NAMEREF{drop-first-n}
            (  \VAR{N}', 
                   \VAR{V}\STAR )
      }
\\
  \KEY{Rule} \quad
    & \NAMEREF{drop-first-n}
        (  \VAR{N} : \NAMEHYPER{../../Primitive}{Integers}{positive-integers}, 
               (   \  ) ) \leadsto 
        (   \  )
\end{align*}
\begin{align*}
  \KEY{Funcon} \quad
  & \NAMEDECL{reverse}(
                       \_ : (  \VAR{T} )\STAR) 
    :  \TO (  \VAR{T} )\STAR 
\\
  \KEY{Rule} \quad
    & \NAMEREF{reverse}
        (   \  ) \leadsto 
        (   \  )
\\
  \KEY{Rule} \quad
    & \NAMEREF{reverse}
        (  \VAR{V} : \VAR{T}, 
               \VAR{V}\STAR : (  \VAR{T} )\STAR ) \leadsto 
        (  \NAMEREF{reverse}
                (  \VAR{V}\STAR ), 
               \VAR{V} )
\end{align*}
\begin{align*}
  \KEY{Funcon} \quad
  & \NAMEDECL{n-of}(
                       \VAR{N} : \NAMEHYPER{../../Primitive}{Integers}{natural-numbers}, \VAR{V} : \VAR{T}) 
    :  \TO (  \VAR{T} )\STAR 
\\
  \KEY{Rule} \quad
    & \NAMEREF{n-of}
        (  0, 
               \_ : \VAR{T} ) \leadsto 
        (   \  )
\\
  \KEY{Rule} \quad
    & \RULE{
      & \NAMEHYPER{../../Primitive}{Integers}{natural-predecessor}
          (  \VAR{N} ) \leadsto 
          \VAR{N}'
      }{
      & \NAMEREF{n-of}
          (  \VAR{N} : \NAMEHYPER{../../Primitive}{Integers}{positive-integers}, 
                 \VAR{V} : \VAR{T} ) \leadsto 
          (  \VAR{V}, 
                 \NAMEREF{n-of}
                  (  \VAR{N}', 
                         \VAR{V} ) )
      }
\end{align*}
\begin{align*}
  \KEY{Funcon} \quad
  & \NAMEDECL{intersperse}(
                       \_ : \VAR{T}', \_ : (  \VAR{T} )\STAR) 
    :  \TO (  \VAR{T}, 
                          (  \VAR{T}', 
                                \VAR{T} )\STAR )\QUERY 
\\
  \KEY{Rule} \quad
    & \NAMEREF{intersperse}
        (  \_ : \VAR{T}', 
               (   \  ) ) \leadsto 
        (   \  )
\\
  \KEY{Rule} \quad
    & \NAMEREF{intersperse}
        (  \_ : \VAR{T}', 
               \VAR{V} ) \leadsto 
        \VAR{V}
\\
  \KEY{Rule} \quad
    & \NAMEREF{intersperse}
        (  \VAR{V}' : \VAR{T}', 
               \VAR{V}\SUB{1} : \VAR{T}, 
               \VAR{V}\SUB{2} : \VAR{T}, 
               \VAR{V}\STAR : (  \VAR{T} )\STAR ) \leadsto 
        (  \VAR{V}\SUB{1}, 
               \VAR{V}', 
               \NAMEREF{intersperse}
                (  \VAR{V}', 
                       \VAR{V}\SUB{2}, 
                       \VAR{V}\STAR ) )
\end{align*}
% 


