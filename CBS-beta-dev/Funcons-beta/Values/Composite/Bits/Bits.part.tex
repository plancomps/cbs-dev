% 



    OUTLINE
  \tableofcontents
\begin{center}
\rule{3in}{0.4pt}
\end{center}

\subsubsection{Bits and bit vectors}\hypertarget{bits-and-bit-vectors}{}\label{bits-and-bit-vectors}

\begin{align*}
  [ \
  \KEY{Type} \quad & \NAMEREF{bits} \\
  \KEY{Datatype} \quad & \NAMEREF{bit-vectors} \\
  \KEY{Funcon} \quad & \NAMEREF{bit-vector} \\
  \KEY{Type} \quad & \NAMEREF{bytes} \\
  \KEY{Alias} \quad & \NAMEREF{octets} \\
  \KEY{Funcon} \quad & \NAMEREF{bit-vector-not} \\
  \KEY{Funcon} \quad & \NAMEREF{bit-vector-and} \\
  \KEY{Funcon} \quad & \NAMEREF{bit-vector-or} \\
  \KEY{Funcon} \quad & \NAMEREF{bit-vector-xor} \\
  \KEY{Funcon} \quad & \NAMEREF{bit-vector-shift-left} \\
  \KEY{Funcon} \quad & \NAMEREF{bit-vector-logical-shift-right} \\
  \KEY{Funcon} \quad & \NAMEREF{bit-vector-arithmetic-shift-right} \\
  \KEY{Funcon} \quad & \NAMEREF{integer-to-bit-vector} \\
  \KEY{Funcon} \quad & \NAMEREF{bit-vector-to-integer} \\
  \KEY{Funcon} \quad & \NAMEREF{bit-vector-to-natural} \\
  \KEY{Funcon} \quad & \NAMEREF{unsigned-bit-vector-maximum} \\
  \KEY{Funcon} \quad & \NAMEREF{signed-bit-vector-maximum} \\
  \KEY{Funcon} \quad & \NAMEREF{signed-bit-vector-minimum} \\
  \KEY{Funcon} \quad & \NAMEREF{is-in-signed-bit-vector} \\
  \KEY{Funcon} \quad & \NAMEREF{is-in-unsigned-bit-vector}
  \ ]
\end{align*}
\paragraph{Bits}\hypertarget{bits}{}\label{bits}

\begin{align*}
  \KEY{Type} \quad 
  & \NAMEDECL{bits}  
    \leadsto \NAMEHYPER{../../Primitive}{Booleans}{booleans}
\end{align*}
$\SHADE{\NAMEHYPER{../../Primitive}{Booleans}{false}}$ represents the absence of a bit, $\SHADE{\NAMEHYPER{../../Primitive}{Booleans}{true}}$ its presence.

\paragraph{Bit vectors}\hypertarget{bit-vectors}{}\label{bit-vectors}

\begin{align*}
  \KEY{Datatype} \quad 
  \NAMEDECL{bit-vectors}(
                     \VAR{N} : \NAMEHYPER{../../Primitive}{Integers}{natural-numbers}) 
  \ ::= \ & \NAMEDECL{bit-vector}(
                               \_ : \NAMEREF{bits}^{\VAR{N}})
\end{align*}
\begin{align*}
  \KEY{Type} \quad 
  & \NAMEDECL{bytes}  
    \leadsto \NAMEREF{bit-vectors}
               (  8 )
\\
  \KEY{Alias} \quad
  & \NAMEDECL{octets} = \NAMEREF{bytes}
\end{align*}
\begin{align*}
  \KEY{Meta-variables} \quad
  & \VAR{BT} <: \NAMEREF{bit-vectors}
                                                     (  \_ )
\end{align*}
\begin{align*}
  \KEY{Built-in Funcon} \quad
  & \NAMEDECL{bit-vector-not}(
                       \_ : \VAR{BT}) 
    :  \TO \VAR{BT} 
\end{align*}
\begin{align*}
  \KEY{Built-in Funcon} \quad
  & \NAMEDECL{bit-vector-and}(
                       \_ : \VAR{BT}, \_ : \VAR{BT}) 
    :  \TO \VAR{BT} 
\end{align*}
\begin{align*}
  \KEY{Built-in Funcon} \quad
  & \NAMEDECL{bit-vector-or}(
                       \_ : \VAR{BT}, \_ : \VAR{BT}) 
    :  \TO \VAR{BT} 
\end{align*}
\begin{align*}
  \KEY{Built-in Funcon} \quad
  & \NAMEDECL{bit-vector-xor}(
                       \_ : \VAR{BT}, \_ : \VAR{BT}) 
    :  \TO \VAR{BT} 
\end{align*}
The above four funcons are the natural extensions of funcons from $\SHADE{\NAMEHYPER{../../Primitive}{Booleans}{booleans}}$
  to $\SHADE{\NAMEREF{bit-vectors}
           (  \VAR{N} )}$ of the same length.

\begin{align*}
  \KEY{Built-in Funcon} \quad
  & \NAMEDECL{bit-vector-shift-left}(
                       \_ : \VAR{BT}, \_ : \NAMEHYPER{../../Primitive}{Integers}{natural-numbers}) 
    : \VAR{BT} 
\end{align*}
\begin{align*}
  \KEY{Built-in Funcon} \quad
  & \NAMEDECL{bit-vector-logical-shift-right}(
                       \_ : \VAR{BT}, \_ : \NAMEHYPER{../../Primitive}{Integers}{natural-numbers}) 
    : \VAR{BT} 
\end{align*}
\begin{align*}
  \KEY{Built-in Funcon} \quad
  & \NAMEDECL{bit-vector-arithmetic-shift-right}(
                       \_ : \VAR{BT}, \_ : \NAMEHYPER{../../Primitive}{Integers}{natural-numbers}) 
    : \VAR{BT} 
\end{align*}
\begin{align*}
  \KEY{Built-in Funcon} \quad
  & \NAMEDECL{integer-to-bit-vector}(
                       \_ : \NAMEHYPER{../../Primitive}{Integers}{integers}, \VAR{N} : \NAMEHYPER{../../Primitive}{Integers}{natural-numbers}) 
    : \NAMEREF{bit-vectors}
        (  \VAR{N} ) 
\end{align*}
$\SHADE{\NAMEREF{integer-to-bit-vector}
           (  \VAR{M}, 
                  \VAR{N} )}$ converts an integer $\SHADE{\VAR{M}}$ to a bit-vector of
  length $\SHADE{\VAR{N}}$, using Two's Complement representation.  If the integer is out of
  range of the representation, it will wrap around (modulo 2\^{}N).

\begin{align*}
  \KEY{Built-in Funcon} \quad
  & \NAMEDECL{bit-vector-to-integer}(
                       \_ : \VAR{BT}) 
    :  \TO \NAMEHYPER{../../Primitive}{Integers}{integers} 
\end{align*}
$\SHADE{\NAMEREF{bit-vector-to-integer}
           (  \VAR{B} )}$ interprets a bit-vector $\SHADE{\VAR{BV}}$ as an integer
  in Two's Complement representation.

\begin{align*}
  \KEY{Built-in Funcon} \quad
  & \NAMEDECL{bit-vector-to-natural}(
                       \_ : \VAR{BT}) 
    :  \TO \NAMEHYPER{../../Primitive}{Integers}{natural-numbers} 
\end{align*}
$\SHADE{\NAMEREF{bit-vector-to-natural}
           (  \VAR{BV} )}$ interprets a bit-vector $\SHADE{\VAR{BV}}$ as a natural number
  in unsigned representation.

\begin{align*}
  \KEY{Funcon} \quad
  & \NAMEDECL{unsigned-bit-vector-maximum}(
                       \VAR{N} : \NAMEHYPER{../../Primitive}{Integers}{natural-numbers}) 
    :  \TO \NAMEHYPER{../../Primitive}{Integers}{natural-numbers} \\&\quad
    \leadsto \NAMEHYPER{../../Primitive}{Integers}{integer-subtract}
               (  \NAMEHYPER{../../Primitive}{Integers}{integer-power}
                       (  2, 
                              \VAR{N} ), 
                      1 )
\end{align*}
\begin{align*}
  \KEY{Funcon} \quad
  & \NAMEDECL{signed-bit-vector-maximum}(
                       \VAR{N} : \NAMEHYPER{../../Primitive}{Integers}{natural-numbers}) 
    :  \TO \NAMEHYPER{../../Primitive}{Integers}{integers} \\&\quad
    \leadsto \NAMEHYPER{../../Primitive}{Integers}{integer-subtract}
               (  \NAMEHYPER{../../Primitive}{Integers}{integer-power}
                       (  2, 
                              \NAMEHYPER{../../Primitive}{Integers}{integer-subtract}
                               (  \VAR{N}, 
                                      1 ) ), 
                      1 )
\end{align*}
\begin{align*}
  \KEY{Funcon} \quad
  & \NAMEDECL{signed-bit-vector-minimum}(
                       \VAR{N} : \NAMEHYPER{../../Primitive}{Integers}{natural-numbers}) 
    :  \TO \NAMEHYPER{../../Primitive}{Integers}{integers} \\&\quad
    \leadsto \NAMEHYPER{../../Primitive}{Integers}{integer-negate}
               (  \NAMEHYPER{../../Primitive}{Integers}{integer-power}
                       (  2, 
                              \NAMEHYPER{../../Primitive}{Integers}{integer-subtract}
                               (  \VAR{N}, 
                                      1 ) ) )
\end{align*}
\begin{align*}
  \KEY{Funcon} \quad
  & \NAMEDECL{is-in-signed-bit-vector}(
                       \VAR{M} : \NAMEHYPER{../../Primitive}{Integers}{integers}, \VAR{N} : \NAMEHYPER{../../Primitive}{Integers}{natural-numbers}) 
    :  \TO \NAMEHYPER{../../Primitive}{Booleans}{booleans} \\&\quad
    \leadsto \NAMEHYPER{../../Primitive}{Booleans}{and}
               ( \\&\quad\quad\quad\quad \NAMEHYPER{../../Primitive}{Integers}{integer-is-less-or-equal}
                       (  \VAR{M}, 
                              \NAMEREF{signed-bit-vector-maximum}
                               (  \VAR{N} ) ), \\&\quad\quad\quad\quad
                      \NAMEHYPER{../../Primitive}{Integers}{integer-is-greater-or-equal}
                       (  \VAR{M}, 
                              \NAMEREF{signed-bit-vector-minimum}
                               (  \VAR{N} ) ) )
\end{align*}
\begin{align*}
  \KEY{Funcon} \quad
  & \NAMEDECL{is-in-unsigned-bit-vector}(
                       \VAR{M} : \NAMEHYPER{../../Primitive}{Integers}{integers}, \VAR{N} : \NAMEHYPER{../../Primitive}{Integers}{natural-numbers}) 
    :  \TO \NAMEHYPER{../../Primitive}{Booleans}{booleans} \\&\quad
    \leadsto \NAMEHYPER{../../Primitive}{Booleans}{and}
               ( \\&\quad\quad\quad\quad \NAMEHYPER{../../Primitive}{Integers}{integer-is-less-or-equal}
                       (  \VAR{M}, 
                              \NAMEREF{unsigned-bit-vector-maximum}
                               (  \VAR{N} ) ), \\&\quad\quad\quad\quad
                      \NAMEHYPER{../../Primitive}{Integers}{integer-is-greater-or-equal}
                       (  \VAR{M}, 
                              0 ) )
\end{align*}
% 


